\hypertarget{title-slide}{}
\hypertarget{absorption-costing-problems}{%
\section{Absorption costing
problems}\label{absorption-costing-problems}}

\hypertarget{title-slide}{}
\hypertarget{absorption-costing-problems}{}
\hypertarget{absorption-costing-problems-1}{%
\section{Absorption costing
problems}\label{absorption-costing-problems-1}}

\hypertarget{aspen-view}{%
\subsubsection{10-8 Aspen View}\label{aspen-view}}

Aspen View produces a full line of sunglasses. This year it began
producing a new model of sunglasses, the Peak 32. It produced 5,300
pairs and sold 4,900 pairs. The following table summarizes the fixed and
variable costs of producing Peak 32 sunglasses. Aspen View uses variable
costing to value its ending inventory.

\begin{longtable}[]{@{}llll@{}}
\toprule\noalign{}
& Fixed Cost & Variable Cost & Total Cost \\
\midrule\noalign{}
\endhead
\bottomrule\noalign{}
\endlastfoot
Direct labor & & \$ 3.50 & \$ 3.50 \\
Direct material & & 7.50 & 7.50 \\
Manufacturing overhead & \$3.20 & 4.50 & 7.70 \\
Advertising & 1.20 & 1.70 & 2.90 \\
Distribution & 0.70 & 0.25 & 0.95 \\
Selling & 1.20 & 0.90 & 2.10 \\
Total cost & \$6.30 & \$18.35 & \$24.65 \\
\end{longtable}

\hypertarget{variable-costing-overview}{%
\subsubsection{Variable costing
overview:}\label{variable-costing-overview}}

\begin{itemize}
\tightlist
\item
  \emph{Motivation:} Solve some of the problems with full absorption
  costing.
\item
  Problem 1: Death Spiral (effective)
\item
  Problem 2: Incentive to overproduce (effective, if we correctly
  separate fixed and variable costs).
\end{itemize}

\hypertarget{variable-costing-overview-1}{%
\subsubsection{Variable costing
overview:}\label{variable-costing-overview-1}}

\begin{itemize}
\tightlist
\item
  All fixed costs hit income in the year that they are incurred.
\item
  Fixed costs do not get absorbed into inventory.
\end{itemize}

\hypertarget{q1}{%
\subsubsection{Q1:}\label{q1}}

What is Aspen View's ending inventory value of Peak 32 sunglasses?

\hypertarget{q1-solution}{%
\subsubsection{Q1: Solution}\label{q1-solution}}

Ending inventory value using variable costing:

\begin{longtable}[]{@{}ll@{}}
\toprule\noalign{}
\endhead
\bottomrule\noalign{}
\endlastfoot
Direct labor & \$3.50 \\
Direct material & 7.50 \\
Variable manufacturing overhead & 4.50 \\
Total variable cost of product & \$15.50 \\
Units produced & 5,300 \\
Units sold & 4,900 \\
Ending inventory & 400 \\
{×} Unit manufacturing cost & \$15.50 \\
Ending inventory value & \$6,200 \\
\end{longtable}

\hypertarget{q2}{%
\subsubsection{Q2:}\label{q2}}

Aspen View is considering switching from variable costing to absorption
costing. Would this year's net income from Peak 32 sunglasses be higher
or lower using absorption costing? Explain.

\hypertarget{q2-solution}{%
\subsubsection{Q2: Solution}\label{q2-solution}}

\begin{itemize}
\tightlist
\item
  Income would have been \textbf{higher} had Aspen View used absorption
  costing.
\item
  Under absorption costing, some of the fixed manufacturing costs would
  have been allocated to the ending inventory rather than all of them
  being written off to cost of goods sold.
\end{itemize}

\hypertarget{q3}{%
\subsubsection{Q3:}\label{q3}}

Suppose Aspen View uses absorption costing. If, instead of producing
5,300 pairs of Peak 32s it produced only 5,000, would net income from
Peak 32 sunglasses be higher or lower from the smaller production
compared to the larger production? Explain.

\hypertarget{q3-solution}{%
\subsubsection{Q3: Solution}\label{q3-solution}}

\begin{itemize}
\tightlist
\item
  Assuming constant variable cost per unit, income would have been
  lower.
\item
  With fewer units produced, less fixed costs would have been allocated
  to the ending inventory under absorption costing.
\item
  The preceding statement assumes variable cost per unit is constant.
\end{itemize}

\hypertarget{q4}{%
\subsubsection{Q4:}\label{q4}}

Aspen View has an opportunity cost of capital of 20 percent. What is the
cost of producing 5,300 pairs of Peak 32s instead of 4,900 pairs?

\hypertarget{q4-solution}{%
\subsubsection{Q4: Solution}\label{q4-solution}}

\begin{itemize}
\tightlist
\item
  Assuming that they can sell the 400 pairs of sunglasses in inventory,
  the cost of overproducing is the sum of:

  \begin{enumerate}
  \def\labelenumi{\arabic{enumi}.}
  \tightlist
  \item
    the additional warehousing costs plus
  \item
    {400 × \$15.50 × 20\% × \emph{ξ}} where {\emph{ξ}} is the fraction
    of the year the glasses are held until being sold.
  \end{enumerate}
\end{itemize}

\hypertarget{q4-solution-1}{%
\subsubsection{Q4: Solution}\label{q4-solution-1}}

\begin{itemize}
\tightlist
\item
  This calculation assumes that all of the variable advertising,
  distribution, and selling expenses are incurred when the sunglasses
  are sold, not manufactured.
\item
  This illustrates both the overproduction incentive from full
  absorption costing and the improvement from variable costing.
\end{itemize}

\hypertarget{kothari-inc.}{%
\subsection{Kothari Inc.}\label{kothari-inc.}}

The telecom division of Kothari Inc.~produces and sells 100,000 line
modulators. Half of the modulators are sold externally at \$150 per
unit, and the other half are sold internally at \textbf{variable
manufacturing costs plus 10 percent}. Kothari uses variable costing to
evaluate the telecom division. The following summarizes the cost
structure of the telecom division.

\hypertarget{kothari-inc.-1}{%
\subsection{Kothari Inc.}\label{kothari-inc.-1}}

\begin{longtable}[]{@{}ll@{}}
\toprule\noalign{}
& Var. Cost \\
\midrule\noalign{}
\endhead
\bottomrule\noalign{}
\endlastfoot
Materials & 27.00 \\
Labor & 12.00 \\
Overhead & 4.00 \\
Total manufacturing cost & 43.00 \\
Fixed manufacturing overhead & \$1,700,000 \\
Variable period costs (per units) & \$18.00 \\
Fixed period costs & \$1,900,000 \\
\end{longtable}

\hypertarget{q1-1}{%
\subsubsection{Q1:}\label{q1-1}}

\begin{itemize}
\tightlist
\item
  Calculate the net income of the telecom division (before taxes) using
  variable costing.
\end{itemize}

\hypertarget{q1-solution-revenue}{%
\subsubsection{Q1 Solution (Revenue):}\label{q1-solution-revenue}}

\begin{longtable}[]{@{}ll@{}}
\toprule\noalign{}
\endhead
\bottomrule\noalign{}
\endlastfoot
Revenues: & \\
Internal sales {(50,000×1.1×\$43)} & \$2,365,000 \\
External sales {(50,000×\$150)} & 7,500,000 \\
Total revenue & \$9,865,000 \\
\end{longtable}

\hypertarget{q1-solution-cost}{%
\subsubsection{Q1 Solution (Cost):}\label{q1-solution-cost}}

\begin{longtable}[]{@{}ll@{}}
\toprule\noalign{}
\endhead
\bottomrule\noalign{}
\endlastfoot
Total revenue & \$9,865,000 \\
Less: & \\
Variable manufacturing cost & \$4,300,000 \\
Fixed manufacturing overhead & 1,700,000 \\
Variable period cost & 1,800,000 \\
Fixed period cost & 1,900,000 \\
Net income & \textbf{\$165,000} \\
\end{longtable}

\hypertarget{q2-1}{%
\subsubsection{Q2:}\label{q2-1}}

\begin{itemize}
\tightlist
\item
  Telcom can outsource the final assembly of all 100,000 modulators for
  \$9.00 per modulator. If it does this, it can reduce variable
  manufacturing cost by \$1.00 per unit and fixed manufacturing overhead
  by \$700,000. If the managers of the telecom unit are compensated
  based on telecom's net income before taxes, do you expect them to
  outsource the final assembly of the modulators? Show calculations.
\end{itemize}

\hypertarget{q2-solution-rev}{%
\subsubsection{Q2: Solution (Rev)}\label{q2-solution-rev}}

\begin{longtable}[]{@{}ll@{}}
\toprule\noalign{}
\endhead
\bottomrule\noalign{}
\endlastfoot
Revenues: & \\
Internal sales {(50,000×1.1×\$51)} & \$ 2,805,000 \\
{  \$51 = \$43 + \$9 − \$1} & \\
External sales {(50,000×\$150)} & 7,500,000 \\
Total revenue & \$10,305,000 \\
\end{longtable}

\hypertarget{q2-solution-cost}{%
\subsubsection{Q2: Solution (Cost)}\label{q2-solution-cost}}

\begin{longtable}[]{@{}
  >{\raggedright\arraybackslash}p{(\columnwidth - 2\tabcolsep) * \real{0.7700}}
  >{\raggedright\arraybackslash}p{(\columnwidth - 2\tabcolsep) * \real{0.2000}}@{}}
\toprule\noalign{}
\endhead
\bottomrule\noalign{}
\endlastfoot
Total revenue & \$10,305,000 \\
Less: & \\
Outsourcing {(100,000×\$9)} & \$ 900,000 \\
Variable manufacturing cost {(100,00×(\$43−\$1))} & 4,200,000 \\
Fixed manufacturing overhead & 1,000,000 \\
Variable period cost & 1,800,000 \\
Fixed period cost & 1,900,000 \\
Net income & \$ 505,000 \\
\end{longtable}

\hypertarget{q2-solution-1}{%
\subsubsection{Q2: Solution}\label{q2-solution-1}}

The Telecom managers face a strong incentive to outsource because their
net income increases from \$165,000 to \$505,000.

\hypertarget{q3-1}{%
\subsubsection{Q3:}\label{q3-1}}

\begin{itemize}
\tightlist
\item
  What happens to the net cash flows of Kothari Inc.~if the final
  assembly of the modulators is outsourced?
\end{itemize}

\hypertarget{q4-solution-2}{%
\subsubsection{Q4: Solution}\label{q4-solution-2}}

\begin{longtable}[]{@{}ll@{}}
\toprule\noalign{}
\endhead
\bottomrule\noalign{}
\endlastfoot
Outsourcing costs {(\$9×100,000)} & \$900,000 \\
Savings: & \\
Variable cost {(\$1×100,000)} & -100,000 \\
Fixed manufacturing overhead & -700,000 \\
Net loss from outsourcing & \$100,000 \\
\end{longtable}

\hypertarget{are-there-alternatives}{%
\subsubsection{Are there alternatives?}\label{are-there-alternatives}}

\begin{itemize}
\tightlist
\item
  Simply centralize outsourcing decisions!
\item
  Contract allocation of fixed costs internally (managers must forecast
  and pay no matter what happens in the future).
\item
  Other alternatives?
\end{itemize}

\hypertarget{naviski}{%
\subsection{Naviski}\label{naviski}}

Navisky designs, manufactures, and sells specialized GPS (global
positioning system) devices for commercial applications.

\hypertarget{navisky}{%
\subsubsection{Navisky}\label{navisky}}

\begin{itemize}
\tightlist
\item
  For example, Navisky currently sells a system for environmental
  studies and is planning systems for private aviation and fleet
  management. The firm has a design team that identifies potential
  commercial GPS applications and then designs and develops prototypes.
\item
  Once a prototype is deemed successful and senior management determines
  that a market exists for the new application, the new design is put
  into production, and the firm markets the new product through
  independent salespeople, direct marketing, trade shows, or whatever
  channel is most appropriate for that market.
\end{itemize}

\hypertarget{navisky-1}{%
\subsubsection{Navisky}\label{navisky-1}}

Currently, Navisky has one very successful system in production (for
environmental studies) and several others in development. Navisky,
located in Austria, is one of nine wholly owned subsidiaries of a large
Swiss conglomerate.

\hypertarget{navisky-incentives}{%
\subsubsection{Navisky: Incentives}\label{navisky-incentives}}

\begin{itemize}
\tightlist
\item
  Andreas Hoffman, president of Navisky, expects to retire next year.
\item
  He receives a fixed salary and a bonus based on reported accounting
  earnings.
\item
  The bonus is 5 percent of earnings in excess of €850,000 for actual
  earnings between €850,000 and €1,400,000.
\item
  If actual earnings exceed €1,400,000, the bonus is capped at:
\end{itemize}

{27, 500{[}5\%×(1,400,000−850,000){]}}

\begin{itemize}
\tightlist
\item
  (Earnings, both actual and target, are before taxes.)
\end{itemize}

\hypertarget{navisky-data}{%
\subsubsection{Navisky: Data}\label{navisky-data}}

The following data summarize Navisky's current operations (in euros).

\begin{longtable}[]{@{}
  >{\raggedright\arraybackslash}p{(\columnwidth - 4\tabcolsep) * \real{0.4400}}
  >{\raggedright\arraybackslash}p{(\columnwidth - 4\tabcolsep) * \real{0.2500}}
  >{\raggedright\arraybackslash}p{(\columnwidth - 4\tabcolsep) * \real{0.2700}}@{}}
\toprule\noalign{}
\begin{minipage}[b]{\linewidth}\raggedright
\end{minipage} & \begin{minipage}[b]{\linewidth}\raggedright
Annual Fixed Costs
\end{minipage} & \begin{minipage}[b]{\linewidth}\raggedright
Variable Costs/Unit
\end{minipage} \\
\midrule\noalign{}
\endhead
\bottomrule\noalign{}
\endlastfoot
Development Costs & 900,000 & \\
Selling and administration costs & 1,100,000 & 300 \\
Manufacturing overhead & 2,700,000 & 190 \\
Direct materials & & 140 \\
Manufacturing labor & & 50 \\
Total & 4,700,000 & 680 \\
Selling price unit & 5,500 & \\
\end{longtable}

\hypertarget{navisky-data-1}{%
\subsubsection{Navisky: Data}\label{navisky-data-1}}

Senior management at Navisky, including Mr.~Hoffman, expects to sell
about 1,200 units of the environmental GPS device this year. However,
they have considerable discretion in setting production levels.
\textbf{Their plant has excess capacity and can produce up to 1,500
environmental devices without seeing any increase in the variable
manufacturing costs per unit.}

\hypertarget{navisky-data-2}{%
\subsubsection{Navisky: Data}\label{navisky-data-2}}

Navisky uses a \textbf{traditional absorption costing system} to absorb
manufacturing overhead into product costs for inventory valuation and to
calculate earnings for internal compensation purposes as well as
external reporting. At the beginning of the current fiscal year, there
was no beginning inventory of the environmental GPS devices.

\hypertarget{q1-2}{%
\subsubsection{Q1:}\label{q1-2}}

How many units of the environmental GPS device would Mr.~Hoffman like to
see Navisky produce if he expects to sell 1,200 devices this year?

\hypertarget{q1-solution-1}{%
\subsubsection{Q1: Solution}\label{q1-solution-1}}

\begin{longtable}[]{@{}
  >{\raggedright\arraybackslash}p{(\columnwidth - 8\tabcolsep) * \real{0.3400}}
  >{\raggedright\arraybackslash}p{(\columnwidth - 8\tabcolsep) * \real{0.1500}}
  >{\raggedright\arraybackslash}p{(\columnwidth - 8\tabcolsep) * \real{0.1500}}
  >{\raggedright\arraybackslash}p{(\columnwidth - 8\tabcolsep) * \real{0.1500}}
  >{\raggedright\arraybackslash}p{(\columnwidth - 8\tabcolsep) * \real{0.1500}}@{}}
\toprule\noalign{}
\endhead
\bottomrule\noalign{}
\endlastfoot
Production & 1200 & 1300 & 1350 & 1360 \\
Revenue (assuming sales of 1200 units) & €6,600,000 & €6,600,000 &
€6,600,000 & €6,600,000 \\
Cost of goods sold: & & & & \\
Variable mfg cost & (456,000) & (456,000) & (456,000) & (456,000) \\
Fixed mfg overhead & (2,700,000) & (2,492,308) & (2,400,000) &
(2,382,353) \\
\end{longtable}

\emph{this is the classic absorption costing problem}

\hypertarget{note}{%
\subsubsection{Note:}\label{note}}

You can think of the formula for fixed manufacturing overhead applied to
cost of goods sold as:

{\emph{FMO} = \emph{OHR} × \emph{Q}\textsubscript{\emph{sold}}}

\begin{itemize}
\tightlist
\item
  OHR is the overhead rate:
  {\emph{OHR} = \emph{OH}/\emph{Q}\textsubscript{\emph{made}}}
\item
  OH is the total overhead incurred, 2.7 million in this case
\item
  {\emph{Q}\textsubscript{\emph{made}}} is the number of units produced,
  and {\emph{Q}\textsubscript{\emph{sold}}} is the number of units sold.
\end{itemize}

\hypertarget{q1-solution-2}{%
\subsubsection{Q1: Solution}\label{q1-solution-2}}

\begin{longtable}[]{@{}
  >{\raggedright\arraybackslash}p{(\columnwidth - 8\tabcolsep) * \real{0.3400}}
  >{\raggedright\arraybackslash}p{(\columnwidth - 8\tabcolsep) * \real{0.1500}}
  >{\raggedright\arraybackslash}p{(\columnwidth - 8\tabcolsep) * \real{0.1500}}
  >{\raggedright\arraybackslash}p{(\columnwidth - 8\tabcolsep) * \real{0.1500}}
  >{\raggedright\arraybackslash}p{(\columnwidth - 8\tabcolsep) * \real{0.1500}}@{}}
\toprule\noalign{}
\endhead
\bottomrule\noalign{}
\endlastfoot
Period costs: & & & & \\
Development costs & (900,000) & (900,000) & (900,000) & (900,000) \\
Fixed Selling and administration costs & (1,100,000) & (1,100,000) &
(1,100,000) & (1,100,000) \\
Variable selling and admin costs & (360,000) & (360,000) & (360,000) &
(360,000) \\
Actual earnings before taxes & €1,084,000 & €1,291,692 & €1,384,000 &
€1,401,647 \\
Bonus & €11,700 & €22,084 & €26,700 & €27,500 \\
\end{longtable}

\hypertarget{q1-solution-3}{%
\subsubsection{Q1: Solution}\label{q1-solution-3}}

Mr.~Hoffman, because he expects to retire next year and hence will not
have to deal with any excess inventory, has an \textbf{incentive to over
produce}. The table below indicates that given sales of 1200 units
Mr.~Hoffman would like to produce about 1,360 units. At 1,360 units,
expected earnings are about €1,401,647, or just above the bonus cap of
€1,400,000. So to maximize his bonus, Mr.~Hoffman will want to produce
1,360 units, or 160 more than he expects to sell.

\hypertarget{q2-2}{%
\subsubsection{Q2:}\label{q2-2}}

Suppose Mr.~Hoffman's bonus calculation was \textbf{based on net income
after including a charge for inventory holding} costs at 20 percent of
the ending inventory value. In other words, his bonus is 5 percent of
net income in excess of \$850,000 up to \$1,400,000 where net income
includes a 20 percent inventory holding cost. How many units of the
environmental GPS device would Mr.~Hoffman like to see produced if he
expects to sell 1,200 devices this year?

\hypertarget{q2-solution-2}{%
\subsubsection{Q2: Solution}\label{q2-solution-2}}

\begin{longtable}[]{@{}
  >{\raggedright\arraybackslash}p{(\columnwidth - 8\tabcolsep) * \real{0.3400}}
  >{\raggedright\arraybackslash}p{(\columnwidth - 8\tabcolsep) * \real{0.1500}}
  >{\raggedright\arraybackslash}p{(\columnwidth - 8\tabcolsep) * \real{0.1500}}
  >{\raggedright\arraybackslash}p{(\columnwidth - 8\tabcolsep) * \real{0.1500}}
  >{\raggedright\arraybackslash}p{(\columnwidth - 8\tabcolsep) * \real{0.1500}}@{}}
\toprule\noalign{}
\begin{minipage}[b]{\linewidth}\raggedright
Production
\end{minipage} & \begin{minipage}[b]{\linewidth}\raggedright
1200
\end{minipage} & \begin{minipage}[b]{\linewidth}\raggedright
1350
\end{minipage} & \begin{minipage}[b]{\linewidth}\raggedright
1400
\end{minipage} & \begin{minipage}[b]{\linewidth}\raggedright
1420
\end{minipage} \\
\midrule\noalign{}
\endhead
\bottomrule\noalign{}
\endlastfoot
Revenue (assuming sales of 1200 units) & €6,600,000 & €6,600,000 &
€6,600,000 & €6,600,000 \\
Cost of goods sold: & & & & \\
Variable mfg cost & (456,000) & (456,000) & (456,000) & (456,000) \\
Fixed mfg overhead & (2,700,000) & (2,400,000) & (2,314,286) &
(2,281,690) \\
\end{longtable}

\hypertarget{q2-solution-3}{%
\subsubsection{Q2: Solution}\label{q2-solution-3}}

\begin{longtable}[]{@{}
  >{\raggedright\arraybackslash}p{(\columnwidth - 8\tabcolsep) * \real{0.3400}}
  >{\raggedright\arraybackslash}p{(\columnwidth - 8\tabcolsep) * \real{0.1500}}
  >{\raggedright\arraybackslash}p{(\columnwidth - 8\tabcolsep) * \real{0.1500}}
  >{\raggedright\arraybackslash}p{(\columnwidth - 8\tabcolsep) * \real{0.1500}}
  >{\raggedright\arraybackslash}p{(\columnwidth - 8\tabcolsep) * \real{0.1500}}@{}}
\toprule\noalign{}
\begin{minipage}[b]{\linewidth}\raggedright
Production
\end{minipage} & \begin{minipage}[b]{\linewidth}\raggedright
1200
\end{minipage} & \begin{minipage}[b]{\linewidth}\raggedright
1350
\end{minipage} & \begin{minipage}[b]{\linewidth}\raggedright
1400
\end{minipage} & \begin{minipage}[b]{\linewidth}\raggedright
1420
\end{minipage} \\
\midrule\noalign{}
\endhead
\bottomrule\noalign{}
\endlastfoot
Period costs: & & & & \\
Development costs & (900,000) & (900,000) & (900,000) & (900,000) \\
Fixed Selling and administration costs & (1,100,000) & (1,100,000) &
(1,100,000) & (1,100,000) \\
Variable selling and admin costs & (360,000) & (360,000) & (360,000) &
(360,000) \\
Actual earnings before inventory costs & €1,084,000 & €1,384,000 &
€1,469,714 & €1,502,310 \\
\end{longtable}

\hypertarget{q2-solution-4}{%
\subsubsection{Q2: Solution}\label{q2-solution-4}}

\begin{longtable}[]{@{}
  >{\raggedright\arraybackslash}p{(\columnwidth - 8\tabcolsep) * \real{0.3500}}
  >{\raggedright\arraybackslash}p{(\columnwidth - 8\tabcolsep) * \real{0.1500}}
  >{\raggedright\arraybackslash}p{(\columnwidth - 8\tabcolsep) * \real{0.1500}}
  >{\raggedright\arraybackslash}p{(\columnwidth - 8\tabcolsep) * \real{0.1500}}
  >{\raggedright\arraybackslash}p{(\columnwidth - 8\tabcolsep) * \real{0.1500}}@{}}
\toprule\noalign{}
\begin{minipage}[b]{\linewidth}\raggedright
Production
\end{minipage} & \begin{minipage}[b]{\linewidth}\raggedright
1200
\end{minipage} & \begin{minipage}[b]{\linewidth}\raggedright
1350
\end{minipage} & \begin{minipage}[b]{\linewidth}\raggedright
1400
\end{minipage} & \begin{minipage}[b]{\linewidth}\raggedright
1420
\end{minipage} \\
\midrule\noalign{}
\endhead
\bottomrule\noalign{}
\endlastfoot
Ending inventory & 0 & 150 & 200 & 220 \\
Cost per unit of inventory & €2630 & €2380 & €2309 & €2281 \\
Ending inventory cost & 0 & 357,000 & 461,800 & 501,820 \\
Weighted average cost of capital & 0.2 & 0.2 & 0.2 & 0.2 \\
(\emph{holding charge}) & & & & \\
Holding cost of inventory & 0 & (71,400) & (92,360) & (100,364) \\
Earnings after inventory cost & €1,084,000 & €1,312,600 & €1,377,354 &
€1,401,946 \\
Bonus & €11,700 & €23,130 & €26,368 & €27,500 \\
\end{longtable}

\hypertarget{q2-solution-5}{%
\subsubsection{Q2: Solution}\label{q2-solution-5}}

With an inventory holding cost of 20 percent deducted from earnings, Mr.
Hoffman will prefer to produce 1,420 units because at this production
level (and given sales of 1,200 units) Mr.~Hoffman will reach the bonus
cap of €27,500.

\hypertarget{q3-2}{%
\subsubsection{Q3:}\label{q3-2}}

Explain why your answers in parts (a) and (b) differ, if they do.

\hypertarget{q3-solution-1}{%
\subsubsection{Q3: Solution}\label{q3-solution-1}}

\begin{itemize}
\tightlist
\item
  Interestingly, charging Mr.~Hoffman an inventory holding cost of 20
  percent actually causes him to over produce even more.
\item
  Without the 20 percent inventory charge Mr.~Hoffman only has to
  produce about 1,360 units (or 160 more than he expects to sell) to
  reach the €1.4 million earnings cap.
\item
  But with the 20 percent inventory charge, Mr.~Hoffman needs to produce
  about 1,420 (or 220 more than he expects to sell) to reach the cap.
\end{itemize}

\hypertarget{q3-solution-2}{%
\subsubsection{Q3: Solution}\label{q3-solution-2}}

\begin{itemize}
\tightlist
\item
  Hence, including the inventory holding charge has the perverse
  incentive of actually causing Mr.~Hoffman to over produce even more.
\item
  The reason for this is the existence of the bonus cap, and the fact
  that the 20 percent charge on inventory is less than the reduction in
  average fixed costs charged to cost of goods sold.
\end{itemize}

\hypertarget{q4-1}{%
\subsubsection{Q4:}\label{q4-1}}

How many units of the environmental GPS device would Mr.~Hoffman like to
see produced, assuming he expects to sell 1,200 devices this year if
Navisky's net income is calculated using variable costing and net income
includes a 20 percent inventory holding cost?

\hypertarget{q4-solution-3}{%
\subsubsection{Q4: Solution}\label{q4-solution-3}}

\begin{longtable}[]{@{}
  >{\raggedright\arraybackslash}p{(\columnwidth - 8\tabcolsep) * \real{0.3400}}
  >{\raggedright\arraybackslash}p{(\columnwidth - 8\tabcolsep) * \real{0.1500}}
  >{\raggedright\arraybackslash}p{(\columnwidth - 8\tabcolsep) * \real{0.1500}}
  >{\raggedright\arraybackslash}p{(\columnwidth - 8\tabcolsep) * \real{0.1500}}
  >{\raggedright\arraybackslash}p{(\columnwidth - 8\tabcolsep) * \real{0.1500}}@{}}
\toprule\noalign{}
\begin{minipage}[b]{\linewidth}\raggedright
Production
\end{minipage} & \begin{minipage}[b]{\linewidth}\raggedright
1200
\end{minipage} & \begin{minipage}[b]{\linewidth}\raggedright
1350
\end{minipage} & \begin{minipage}[b]{\linewidth}\raggedright
1390
\end{minipage} & \begin{minipage}[b]{\linewidth}\raggedright
1400
\end{minipage} \\
\midrule\noalign{}
\endhead
\bottomrule\noalign{}
\endlastfoot
Revenue (assuming sales of 1200 units) & €6,600,000 & €6,600,000 &
€6,600,000 & €6,600,000 \\
Cost of goods sold: & & & & \\
Variable mfg cost & (456,000) & (456,000) & (456,000) & (456,000) \\
Fixed mfg overhead & (2,700,000) & (2,700,000) & (2,700,000) &
(2,700,000) \\
\end{longtable}

\hypertarget{q4-solution-4}{%
\subsubsection{Q4: Solution}\label{q4-solution-4}}

\begin{longtable}[]{@{}
  >{\raggedright\arraybackslash}p{(\columnwidth - 8\tabcolsep) * \real{0.3400}}
  >{\raggedright\arraybackslash}p{(\columnwidth - 8\tabcolsep) * \real{0.1500}}
  >{\raggedright\arraybackslash}p{(\columnwidth - 8\tabcolsep) * \real{0.1500}}
  >{\raggedright\arraybackslash}p{(\columnwidth - 8\tabcolsep) * \real{0.1500}}
  >{\raggedright\arraybackslash}p{(\columnwidth - 8\tabcolsep) * \real{0.1500}}@{}}
\toprule\noalign{}
\begin{minipage}[b]{\linewidth}\raggedright
Production
\end{minipage} & \begin{minipage}[b]{\linewidth}\raggedright
1200
\end{minipage} & \begin{minipage}[b]{\linewidth}\raggedright
1350
\end{minipage} & \begin{minipage}[b]{\linewidth}\raggedright
1390
\end{minipage} & \begin{minipage}[b]{\linewidth}\raggedright
1400
\end{minipage} \\
\midrule\noalign{}
\endhead
\bottomrule\noalign{}
\endlastfoot
Period costs: & & & & \\
Development costs & (900,000) & (900,000) & (900,000) & (900,000) \\
Fixed Selling and administration costs & (1,100,000) & (1,100,000) &
(1,100,000) & (1,100,000) \\
Variable selling and admin costs & (360,000) & (360,000) & (360,000) &
(360,000) \\
Actual earnings before inventory cost & €1,084,000 & €1,084,000 &
€1,084,000 & €1,084,000 \\
\end{longtable}

\hypertarget{q4-solution-5}{%
\subsubsection{Q4: Solution}\label{q4-solution-5}}

\begin{longtable}[]{@{}lllll@{}}
\toprule\noalign{}
Production & 1200 & 1350 & 1390 & 1400 \\
\midrule\noalign{}
\endhead
\bottomrule\noalign{}
\endlastfoot
Ending inventory & 0 & 150 & 190 & 200 \\
Cost per unit of inventory & 380 & 380 & 380 & 380 \\
Ending inventory cost & 0 & 57,000 & 72,200 & 76,000 \\
\end{longtable}

\hypertarget{q4-solution-6}{%
\subsubsection{Q4: Solution}\label{q4-solution-6}}

\begin{longtable}[]{@{}
  >{\raggedright\arraybackslash}p{(\columnwidth - 8\tabcolsep) * \real{0.3500}}
  >{\raggedright\arraybackslash}p{(\columnwidth - 8\tabcolsep) * \real{0.1500}}
  >{\raggedright\arraybackslash}p{(\columnwidth - 8\tabcolsep) * \real{0.1500}}
  >{\raggedright\arraybackslash}p{(\columnwidth - 8\tabcolsep) * \real{0.1500}}
  >{\raggedright\arraybackslash}p{(\columnwidth - 8\tabcolsep) * \real{0.1500}}@{}}
\toprule\noalign{}
\begin{minipage}[b]{\linewidth}\raggedright
Production
\end{minipage} & \begin{minipage}[b]{\linewidth}\raggedright
1200
\end{minipage} & \begin{minipage}[b]{\linewidth}\raggedright
1350
\end{minipage} & \begin{minipage}[b]{\linewidth}\raggedright
1390
\end{minipage} & \begin{minipage}[b]{\linewidth}\raggedright
1400
\end{minipage} \\
\midrule\noalign{}
\endhead
\bottomrule\noalign{}
\endlastfoot
Weighted average cost of capital & 0.2 & 0.2 & 0.2 & 0.2 \\
Holding cost of inventory & 0 & (11,400) & (14,440) & (15,200) \\
Earnings after inventory cost & €1,084,000 & €1,072,600 & €1,069,560 &
€1,068,800 \\
Bonus & €11,700 & €11,130 & €10,978 & €10,940 \\
\end{longtable}

\hypertarget{q4-solution-7}{%
\subsubsection{Q4: Solution}\label{q4-solution-7}}

Under variable costing and a 20 percent inventory holding cost,
Mr.~Hoffman will not over produce. He will produce exactly what he
intends to sell, 1,200 devices. If he over produces under variable
costing, earnings falls, and hence his bonus is lower.

\hypertarget{looking-backward-looking-forward.}{%
\subsection{Looking backward, looking
forward.}\label{looking-backward-looking-forward.}}

\begin{itemize}
\tightlist
\item
  We are in the final push to the end of the semester now.
\item
  There are just four lectures with content and one in-class review
  session left.
\end{itemize}

\hypertarget{looking-backward-looking-forward.-1}{%
\subsection{Looking backward, looking
forward.}\label{looking-backward-looking-forward.-1}}

\begin{itemize}
\tightlist
\item
  We may look back on this semester as an inflection point in the
  history of education for two reasons:
\end{itemize}

\begin{enumerate}
\def\labelenumi{\arabic{enumi}.}
\tightlist
\item
  Return to in-person teaching.
\item
  Beginning the A.I. Conversation.
\end{enumerate}

\begin{itemize}
\tightlist
\item
  So I want to start the talk today by looking back at what I think is
  important.
\end{itemize}

\hypertarget{first-simplicity.}{%
\subsection{First, Simplicity.}\label{first-simplicity.}}

A topic that we have not covered explicitly in this course.

\begin{enumerate}
\def\labelenumi{\arabic{enumi}.}
\tightlist
\item
  Simple systems, that you can clearly introspect and explain are
  critical for decision making.
\item
  Complexity is not a virtue, but it is a good way to hide from
  responsibility!
\item
  Our task is to find simple expression of complex systems.
\end{enumerate}

\hypertarget{second-transparency.}{%
\subsection{Second, Transparency.}\label{second-transparency.}}

\begin{enumerate}
\def\labelenumi{\arabic{enumi}.}
\tightlist
\item
  Being able to show your work is, at this stage, a form of
  communication. To your future self and to others who may consume your
  analyses after the fact.
\item
  If you understand the system you will have some understanding of what
  happens when you change the inputs.
\item
  Mistrust opaque systems (this is my most devastating critique of AI)
  understanding why a decision process came to the conclusion that it
  did is as important as understanding the conclusion.
\item
  If you want to find a bad decision, just look for the ones that are
  least well understood.
\end{enumerate}

\hypertarget{third-doubt.}{%
\subsection{Third, Doubt.}\label{third-doubt.}}

This we have talked about at every turn.

Always ask: 1. how is the decision process tricking us? 2. How are the
data tricking us? 3. What is the real goal, what are the data trying to
tell us? 4. What is the goal of the people who made the data?

\hypertarget{third-doubt.-1}{%
\subsection{Third, Doubt.}\label{third-doubt.-1}}

\begin{itemize}
\tightlist
\item
  Are we inheriting the system that gives us this answer form another
  process that has different needs and thus might trick us?
\item
  This is the point of the current discussion of decision making.
\end{itemize}

\hypertarget{fourth-the-answers-are-easy-its-the-questions-that-are-hard.}{%
\subsection{Fourth, The answers are easy, it's the questions that are
hard.}\label{fourth-the-answers-are-easy-its-the-questions-that-are-hard.}}

\begin{itemize}
\tightlist
\item
  You've been trained to answer questions, but that is the easy part,
  the part that can be automated.
\item
  Programming easy things is often about as hard as programming hard
  things. Remember systems of equations in Python!
\item
  Deciding what information to use and how to structure your question
  has always been the most important issue. Even more so now, as we can
  often trivially solve any well structured problem.
\end{itemize}

\hypertarget{fourth-the-answers-are-easy-its-the-questions-that-are-hard.-1}{%
\subsection{Fourth, The answers are easy, it's the questions that are
hard.}\label{fourth-the-answers-are-easy-its-the-questions-that-are-hard.-1}}

\begin{itemize}
\tightlist
\item
  We have inherited an education system from a world where computers
  were people (and not nearly as well paid as they should have been).
\item
  We (educators) used ability to solve complex problems as a measure of
  ability.
\item
  You will need to unlearn complexity, this is something that we all get
  trained for incorrectly.
\end{itemize}

\emph{The minds that did that work and that will do the work you need to
do are not different, but the training, and habits of mind that that you
need in order to succeed at it are very different.}

:::
